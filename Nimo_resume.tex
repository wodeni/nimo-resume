%%%%%%%%%%%%%%%%%%%%%%%%%%%%%%%%%%%%%%%
% Deedy - One Page Two Column Resume
% LaTeX Template
% Version 1.1 (30/4/2014)
%
% Original author:
% Debarghya Das (http://debarghyadas.com)
%
% Original repository:
% https://github.com/deedydas/Deedy-Resume
%
% IMPORTANT: THIS TEMPLATE NEEDS TO BE COMPILED WITH XeLaTeX
%
% This template uses several fonts not included with Windows/Linux by
% default. If you get compilation errors saying a font is missing, find the line
% on which the font is used and either change it to a font included with your
% operating system or comment the line out to use the default font.
%
%%%%%%%%%%%%%%%%%%%%%%%%%%%%%%%%%%%%%%
%
% TODO:
% 1. Integrate biber/bibtex for article citation under publications.
% 2. Figure out a smoother way for the document to flow onto the next page.
% 3. Add styling information for a "Projects/Hacks" section.
% 4. Add location/address information
% 5. Merge OpenFont and MacFonts as a single sty with options.
%
%%%%%%%%%%%%%%%%%%%%%%%%%%%%%%%%%%%%%%
%
% CHANGELOG:
% v1.1:
% 1. Fixed several compilation bugs with \renewcommand
% 2. Got Open-source fonts (Windows/Linux support)
% 3. Added Last Updated
% 4. Move Title styling into .sty
% 5. Commented .sty file.
%
%%%%%%%%%%%%%%%%%%%%%%%%%%%%%%%%%%%%%%%
%
% Known Issues:
% 1. Overflows onto second page if any column's contents are more than the
% vertical limit
% 2. Hacky space on the first bullet point on the second column.
%
%%%%%%%%%%%%%%%%%%%%%%%%%%%%%%%%%%%%%%

\documentclass[]{deedy-resume-openfont}


\begin{document}

%%%%%%%%%%%%%%%%%%%%%%%%%%%%%%%%%%%%%%
%
%     LAST UPDATED DATE
%
%%%%%%%%%%%%%%%%%%%%%%%%%%%%%%%%%%%%%%
% \lastupdated

%%%%%%%%%%%%%%%%%%%%%%%%%%%%%%%%%%%%%%
%
%     TITLE NAME
%
%%%%%%%%%%%%%%%%%%%%%%%%%%%%%%%%%%%%%%

\namesection{Wode ``Nimo''}{Ni}{
\href{mailto:wn2155@columbia.edu}{wn2155@columbia.edu} | 717.218.4574 \\
\urlstyle{same}\url{http://columbia.edu/~wn2155}
% | GitHub: \href{https://github.com/wodeni/}{wodeni}\\
}

%%%%%%%%%%%%%%%%%%%%%%%%%%%%%%%%%%%%%%
%
%     COLUMN ONE
%
%%%%%%%%%%%%%%%%%%%%%%%%%%%%%%%%%%%%%%

\begin{minipage}[t]{0.33\textwidth}

%%%%%%%%%%%%%%%%%%%%%%%%%%%%%%%%%%%%%%
%     EDUCATION
%%%%%%%%%%%%%%%%%%%%%%%%%%%%%%%%%%%%%%

\section{Education}

\subsection{Columbia University}
\descript{B.S. in Computer Science}
\location{2016.09 - present \\New York, NY}
\location{Major GPA: 4.0 \\ Cum. GPA: 3,94}
Vision, Graphics Track \\
Tau Beta Pi \\
Dean's list \\
\sectionsep

\subsection{Dickinson College}
\descript{B.S. in Computer Science}
\location{2013.09 - 2016.05 \\ Carlisle, PA}
\location{Major GPA: 4.0\\
Cum. GPA: 3.93}
Departmental Honor \\
% Summa cum laude \\
Dean's list \\
John Montgomery Scholarship \\
Pi Mu Epsilon \\
Upsilon Pi Epsilon \\
Alpha Lambda Delta \\
\sectionsep


%%%%%%%%%%%%%%%%%%%%%%%%%%%%%%%%%%%%%%
%     COURSEWORK
%%%%%%%%%%%%%%%%%%%%%%%%%%%%%%%%%%%%%%

\section{Coursework}
% \subsection{Graduate}
% \sectionsep

% \subsection{Undergraduate}
Compiler \\
Computer Graphics \\
Computer Vision \\
Robotics \\
Artificial Intelligence \\
Operating Systems  \\
Database Systems \\
Computer Networks \\
Programming Languages \\
Theory of Computation \\
Computer Architecture \\
Data Structures \\
\sectionsep

%%%%%%%%%%%%%%%%%%%%%%%%%%%%%%%%%%%%%%
%     SKILLS
%%%%%%%%%%%%%%%%%%%%%%%%%%%%%%%%%%%%%%

\section{Skills}
\subsection{Programming}
% \location{Over 5000 lines:}
Java    \textbullet{}   C    \textbullet{} C++ \textbullet{} Haskell \\
OCaml \textbullet{} SQL    \textbullet{} Python \\
\LaTeX\ \textbullet{} Scheme \textbullet{} Common Lisp \\
% \location{Over 1000 lines:}
\sectionsep
\subsection{Tools}
Git/GitHub \textbullet{} Unix \textbullet{} Vim\\  Eclipse \textbullet{} XCode \textbullet{} Visual Studio\\
Android Studio \textbullet{} Make/CMake \\
% \sectionsep
% \subsection{Languages}
% \textbullet{} Native: Chinese \\
% \textbullet{} Proficient: English \\
% \textbullet{} Elementary: Japanese \\
% \sectionsep

%%%%%%%%%%%%%%%%%%%%%%%%%%%%%%%%%%%%%%
%     LINKS
%%%%%%%%%%%%%%%%%%%%%%%%%%%%%%%%%%%%%%

\section{Links}
Github:// \href{https://github.com/wodeni}{\custombold{wodeni}} \\
LinkedIn://  \href{https://www.linkedin.com/in/wode-ni}{\custombold{wode-ni}} \\
% YouTube://  \href{https://www.youtube.com/user/DeedyDash007}{\custombold{DeedyDash007}} \\
% Twitter://  \href{https://twitter.com/debarghya_das}{\custombold{@debarghya\_das}} \\
% Quora://  \href{https://www.quora.com/Debarghya-Das}{\custombold{Debarghya-Das}}
% \sectionsep



%%%%%%%%%%%%%%%%%%%%%%%%%%%%%%%%%%%%%%
%
%     COLUMN TWO
%
%%%%%%%%%%%%%%%%%%%%%%%%%%%%%%%%%%%%%%

\end{minipage}
\hfill
\begin{minipage}[t]{0.66\textwidth}


%%%%%%%%%%%%%%%%%%%%%%%%%%%%%%%%%%%%%%
%     Research
%%%%%%%%%%%%%%%%%%%%%%%%%%%%%%%%%%%%%%

\section{Research}
\runsubsection{Penrose}
\descript{| Research at Carnegie Mellon University}
\location{May 2017 - August 2017 | Pittsburgh, PA}
    Penrose is a system that automatically visualizes mathematics.  The system, comprised of two domain-specific languages
    % : Substance models mathematical notations, and Style specifies the visual semantics of Substance, allowing
    , allows users to create professional diagrams by simply typing mathematical notations. I \textbf{designed and implemented the Style language}, and extended the Substance language to support functions and logically quantified statements. The work was presented at  \href{https://conf.researchr.org/event/dsldi-2017/dsldi-2017-substance-and-style-domain-specific-languages-for-mathematical-diagrams}{DSLDI 2017}, co-located with SPLASH 2017. \textit{(Co-advised by \textbf{\href{https://www.cs.cmu.edu/~./aldrich/}{Jonathan Aldrich}}, \textbf{\href{https://www.cs.cmu.edu/~kmcrane/}{Keenan Crane}}, \textbf{\href{http://www.cs.cmu.edu/~jssunshi/}{Joshua Sunshine}}, and \textbf{\href{https://www.cs.cmu.edu/~kqy/}{Katherine Ye}})}

% See official site \href{http://penrose.ink/}{http://penrose.ink/} for more.
\sectionsep

\runsubsection{Cyber Affordance Visualization in Augmented Reality (CAVIAR)}
\descript{| Research at Columbia University}
\location{Jan 2017 - May 2017 | New York, NY}
Supervised by \textbf{\href{http://www.cs.columbia.edu/~feiner/}prof. Steven Feiner}, I participated in the CAVIAR project, in which we built an AR application that visualizes cyber affordance in indoor and outdoor environments. Learned Unity and Hololens development and investigated the construction of 3D models from GIS data.
\sectionsep

\runsubsection{\href{http://scholar.dickinson.edu/student_honors/221/}{Whiteboard Scanning Using Super-Resolution}}
\descript{| Honors Thesis}
\location{May 2016 | Carlisle, PA}
 Supervised by \textbf{\href{http://users.dickinson.edu/~jmac/}{Prof John MacCormick}}(Advisor), \textbf{\href{http://users.dickinson.edu/~wahlst/}{Prof Timothy Wahls}}, and \textbf{\href{http://users.dickinson.edu/~braught/?}{Prof Grant Braught}}, I studied an application of a super-resolution algorithm: to compute a clear, scanned output given a low-quality video of a whiteboard. The work was presented on \textbf{\href{http://ccscne.org/conferences/ccscne-2016/}{CCSCNE 2016}} conference.
\sectionsep


%%%%%%%%%%%%%%%%%%%%%%%%%%%%%%%%%%%%%%
%     Projects
%%%%%%%%%%%%%%%%%%%%%%%%%%%%%%%%%%%%%%

\section{Projects}

\runsubsection{\href{https://github.com/wodeni/MPL}{Matrix Processing Language}}
\descript{|
Designing a new programming language}
% May. 2016}
% In-course Project}
% Compiler and Programming Language Design }
% \location{Dec. 2016|
% | New York, NY}
Designed and implemented a new domain specific language that focuses on matrix computations. Using MPL, we bulit a simulation of Conway's Game of Life in less than 30 lines.
\sectionsep

\runsubsection{\href{https://github.com/wodeni/raytra}{Raytra}}
\descript{|
A ray tracer from scratch}
% Dec. 2016}
% Computer Graphics }
% \location{Dec. 2016 | New York, NY}
Implemented a ray tracer from scratch. This renderer employs Monte-Carlo ray tracing and scene-wide acceleration using BVH-tree. The output images have Blinn-Phong shading, reflections, refraction, and soft shadows.
\sectionsep

% \runsubsection{Building a web server}
% \descript{| In-course Project}
% \location{Dec. 2016 | New York, NY}
% Implemented a web server that supports HTTP 1.0 from scratch in C. The server serves static contents as well as dynamic contents served from a simple database server that is also written from scratch.
% \sectionsep

% \runsubsection{Othello AIs}
% \descript{| In-course Project}
% \location{Dec. 2014 | Carlisle, PA}
% Implemented versions of Othello AIs using different strategies. Compared the effect of these features.
% \sectionsep


%%%%%%%%%%%%%%%%%%%%%%%%%%%%%%%%%%%%%%
%     WORK
%%%%%%%%%%%%%%%%%%%%%%%%%%%%%%%%%%%%%%

\section{Work}

\runsubsection{Columbia Univeristy Computer Science Department}
\descript{ | Teaching Assistant }
\location{September 2017 – Current | New York, NY}
TA for COMS 4115: Programming Languages and Translators, taught by prof. Stephen Edwards.
\sectionsep

% \runsubsection{AsiaInfo}
% \descript{| Summer Intern}
% \location{May 2015 | Guangzhou, China}
% % \vspace{\topsep} % Hacky fix for awkward extra vertical space
% Participated in developing a web application that manages records of servers and applications.
% The system utilizes Struts and Spring frameworks, MyBatis, and Oracle Database.
% \sectionsep

\runsubsection{Dickinson College Computer Science Department}
\descript{ | Teaching Assistant and Lab Consultant }
\location{September 2014 – May 2016 | Carlisle, PA}
TA for Introduction to Java II with Prof. Timothy Wahls during Spring 2016 Semester. Held evening help room sessions to assist students with homework and projects.
\sectionsep

%%%%%%%%%%%%%%%%%%%%%%%%%%%%%%%%%%%%%%
%     AWARDS
%%%%%%%%%%%%%%%%%%%%%%%%%%%%%%%%%%%%%%

% \section{Awards}
% \begin{tabular}{rll}
% 2014	     & top 52/2500  & KPCB Engineering Fellow\\
% 2014	     & 2\textsuperscript{nd} most points  & Google Code Jam, Qualification Round\\
% 2014	     & 1\textsuperscript{st}/50  & Microsoft Coding Competition, Cornell\\
% 2013	     & National  & Jump Trading Challenge Finalist\\
% 2013     & 7\textsuperscript{th}/120 & CS 3410 Cache Race Bot Tournament  \\
% 2012     & 2\textsuperscript{nd}/150 & CS 3110 Biannual Intra-Class Bot Tournament \\
% 2011     & National & Indian National Mathematics Olympiad (INMO) Finalist \\
% 2010     & National & Comp. Soc. of India's National Programming Contest\\
% \end{tabular}
% \sectionsep

%%%%%%%%%%%%%%%%%%%%%%%%%%%%%%%%%%%%%%
%     SOCIETIES
%%%%%%%%%%%%%%%%%%%%%%%%%%%%%%%%%%%%%%

% \section{Societies}

% \begin{tabular}{rll}
% 2015   & Pi Mu Epsilon U.S. Honorary National Mathematics Society\\
% 2014   & Upsilon Pi Epsilon Honor Society for the Computing and Information Disciplines\\
% 2013   & Alpha Lambda Delta First-Year Honor Society\\
% \end{tabular}
% \sectionsep

%%%%%%%%%%%%%%%%%%%%%%%%%%%%%%%%%%%%%%
%     PUBLICATIONS
%%%%%%%%%%%%%%%%%%%%%%%%%%%%%%%%%%%%%%

% \section{Publications}
% \renewcommand\refname{\vskip -1.5cm} % Couldn't get this working from the .cls file
% \bibliographystyle{abbrv}
% \bibliography{publications}
% \nocite{*}

\end{minipage}
\end{document}  \documentclass[]{article}
